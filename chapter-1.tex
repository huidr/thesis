\chapter{Knots}

\section{Introduction}

The study of knot invariants is an important topic in knot theory. By Reidemeister's theorem, to check if a function is a knot invariant, we only need to check that the function is invariant under the Reidemeister moves. A knot invariant may not be a complete invariant in that two non-equivalent knots may give the same value. There are several invariants established and they vary in their power. For instance, the Jones polynomial is much more powerful than the number of $3$-colorings.

In Section \ref{Alex}, we see Alexander's theorem which builds the connection between knots and braids. We discuss the original proof given by Alexander and its shortcomings. Then we discuss another proof first given by Yamada and later improved by Vogel, which also has some interesting corollaries. In Section \ref{Markov}, we see Markov's theorem and this will give us far more tools to study knots and links from braid theory. In particular, we will construct Markov functions which we can use to construct knot invariants.

In Section \ref{Burau}, we see the Burau representation of braid groups. We will show that the Burau representation is reducible and use the reduced Burau representation to construct a link invariant called the Alexander-Conway polynomial. This is one of the earliest polynomial invariants discovered. Then in Section \ref{Temperley}, we describe the Temperley-Lieb (TL) algebra and construct a bracket polynomial for knots and links. This bracket polynomial is indeed the Kauffman bracket and we will see that a suitably normalized version of this bracket yields the Jones polynomial.

\section{Invariants for Knots and Links}

\section{Alternating Links}

\section{Polynomial Invariants}
