\chapter{Introduction}
\label{chapter1}

In knot theory, the study of invariants is of fundamental importance as they help in detecting non-isotopic knots. An invariant $i : \mathcal{L}/(\cong) \to E $ is a map from the set of links (union of disjoint knots) under equivalence of isotopy to some set $E$. The invariant $i$ is said to be a complete invariant if it is injective. For a knot $K$, the knot group $\pi_1(\R^3-K)$ is known to be a complete invariant.

The object of this thesis to study the Jones polynomial $V_L(t)$, its applications and generalization to Khovanov homology. For a pleasant ride, it is useful to adopt a convenient set of notations and be consistent throughout. We denote Reidemeister moves by $\mathcal{R}^{1,2,3}$. For instance, the bracket polynomial $\left< \cdot \right>$ is $\mathcal{R}^{2,3}$-invariant but gets multiplied by $A^{\pm 3}$ under $\mathcal{R}^1$. A link $L$ has planar projection or diagram $D$. We use $L$ and $D$ interchangeably if there is no chance of confusion. %If $L$ is oriented, we sometimes use $\vec{L}$.

In chapter \ref{chapter2}, we look at braid groups $B_n$ and representations $B_n \to \text{Aut}(F^n)$. One may look at $B_n$ as $\pi_1(\text{UConf}_n(\R^2))$, the fundamental group of the $n$th configuration space (defined later) of $\R^2$, or define it purely algebraically by giving a presentation $\left< S : R \right>$. The presentation has its advantage of an easy transition to braid representations. In particular, we study the Burau representation $\psi_n : B_n \to \text{GL}_n(\Z[t, t^{-1}])$ and its properties. We derive the Alexander polynomial from $\psi_n$, thereby giving the earliest known polynomial invariant its deserved spot in our the study of $V_L(t)$. Then we look at the Temperley-Lieb algebra and use to arrive at the bracket $\left< \cdot \right>$. As we will see, $\left< \cdot \right>$ leads us directly to $V_L(t)$ rather easily.

In chapter \ref{chapter3}, we define $V_L(t)$ using its own skein relation which we use to derive some properties of $V_L(t)$. In general, $V_L(t)$ depends on the orientation of the components of $L$. We will see how we can control this thereby equipping ourselves with a variant of $V_L(t)$ which is independent of orientation. One may generalize $V_L(t)$ to two two-variable polynomials known as the HOMFLYPT polynomial and the Kauffman polynomial. One of the greatest triumphs of $V_L(t)$, though, is in settling some conjectures of classical knot theory and this deserves its own chapter---the next one!

In chapter \ref{chapter4}, we prove the Tait conjectures, which concern reduced alternating links. We make the proofs as general as we can. We use the state-sum model of $\left< \cdot \right>$ and $V_L(t)$ and their properties in the proofs. From these conjectures (now theorems), we deduce several interesting corollaries concerning amphicheiral and links of even crossing number.

In chapter \ref{chapter5}, we look at categorification and motivate it. We recall some concepts from algebraic topology and category theory relevant to our study. We look at homology $H_i(X; R)$ as a categorification of the Euler characteristic $\chi(X)$. The plan is to construct a homology theory such that its graded Euler characteristic is the Jones polynomial---this we discuss in the next chapter.%and how we may obtain a long exact sequence from this homology. We also discuss general philosophy of categorification and decategorification.

In chapter \ref{chapter6}, we construct a (co)homology theory for links following Khovanov's \cite{10.1215/S0012-7094-00-10131-7} and Bar Natan's \cite{10.2140/agt.2002.2.337} approach. This (co)homology is a categorification of $V_L(t)$ and gives us a strictly more powerful invariant than $V_L(t)$. Then we categorify the bracket $\left< \cdot \right>$ following Oleg Viro's \cite{OlegViro2004} approach. %This will lead us to a long exact sequence which provides us an easy way to compute homology groups of torus links $T(2,n)$ for $n > 0$.