\chapter*{Conclusion}

We saw how useful braids are in studying knots. It all started with Alexander's theorem which allows us to transform any knot or link into a closed braid form. Markov's theorem gives us more control in that it guarantees that any two braids whose closure yield ambient isotopic links are M-equivalent and vice versa. These two theorems give us more tools to study knots by use of theory of braids.

The Markov moves are braid analogue of Reidemeister moves for oriented links. This reduces the problem of checking invariance in terms of Reidemeister moves to a problem of checking invariance in terms of Markov moves. Using representation of braid groups and constructing functions which are invariant under Markov moves, we can construct knot or link invariants.

We have seen two such representations: the Burau representation and the Temperley-Lieb algebra. There are other representations which lead to other polynomials and invariants for knots and links. There are some limitations to this approach. For instance, the Burau representation is not faithful for $n\geq 6$. This means the braids which are not equivalent may be mapped to the same element by the representation and as such, the invariants constructed from these representations are not complete invariants for knots.