\chapter{The Jones polynomial}
\label{chapter3}

The Jones polynomial $V_L(t^{1/2})$ is a function from the set of oriented links to the ring $\Z[t^{1/2}, t^{-1/2}]$ of Laurent polynomials. Sometimes by abuse of notation we will write $V_L(t)$ instead of $V_L(t^{1/2})$ as a shorthand for $t = (t^{1/2})^2$. Keeping in mind that the Jones polynomial is defined for oriented links $L$, we use the notations $V_L(t), V_L, V(L)$ to mean the same thing.

One characterizes $V_L(t)$ by giving the value $1$ to any diagram of the unknot (including the unknot diagrams with crossings) and satisfying the skein relation:
\begin{equation}
\label{2:eq:1}
  t^{-1}V_{L_+}(t) - tV(L_-) = (t^{1/2} - t^{-1/2})V(L_0),
\end{equation}
where $L_+$, $L_-$ and $L_0$ are oriented links that differ only locally as shown in the following diagram:

\begin{figure}[h]
  \centering
  \includegraphics[scale=.04]{images/jones-skein-relation.png}
\end{figure}
 
In some sources one finds the Jones polynomial defined in variable $t^{1/4}$. This is not necessary since it never really takes odd powers of $t^{1/4}$. Thus we have defined $V_L$ in variable $t^{1/2}$.

\begin{theorem}
\label{cha:jones-polynomial}
  The Jones polynomial $V_L(t)$ is an invariant for oriented links.
\end{theorem}

One may attempt to prove the above theorem using invariance under $\mathcal{R}^{1,2,3}$. We can also start with the definition Kauffman bracket and obtain the above skein relation. 

\begin{remark}
\label{cha:jones-polynomial-1}
  Suppose we compute $V_L(t)$ for $L_+$. By the above skein relation, we replace $V(L_+)$ by $V(L_-)$ and $V(L_0)$. Thus we are not strictly reducing to lower number of crossings since $V(L_+)$ and $V(L_-)$ has the same number of crossings.
\end{remark}

Although the above skein relation works well for computational purposes, we will use a different skein relation for the rest of our study. Let $L$ be an link. We compute $\langle L \rangle$ using the skein relation:
\begin{align*}
  \left< \KPA \right> &= -A^2 - A^{-2}, \\
  \left< \KPA D \right> &= (-A^2 - A^{-2}) \left< D \right>, \\
  \left< \KPB \right> &= A\left< \KPC \right> + A^{-1}\left< \KPD \right>.
\end{align*}
This polynomial $\left< \cdot \right>$ is called Kauffman bracket polynomial. Recall that we arrived at the bracket polynomial using Temperley-Lieb algebra in the previous chapter. 

\begin{remark}
\label{cha:jones-polynomial-2}
  Comparing the bracket skein relation with the skein relation we used above for $V_L(t)$, we see that while $V_L(t)$ needs orientation the bracket is defined without it.
\end{remark}

\begin{remark}
  Henceforth we will use the notations $L$ and $D$ interchangeably keeping in mind that $D$ is projection of $L$ onto the plane. The author prefers using $D$ and working with planar projections but the Jones polynomial $V_L(t)$ is written using $L$ in most literature.
\end{remark}

\begin{proposition}
\label{cha:jones-polynomial-3}
  The bracket polynomial $\left< \cdot \right>$ is characterized by the three bracket skein relations and invariance under $\mathcal{R}^2$ and $\mathcal{R}^3$.
\end{proposition}

Indeed, one may write down a general skein relation as the following:
\begin{align*}
  \left< \KPA \right> = \delta, \left< \KPA D \right> = \delta \left< D \right>, \left< \KPB \right> = A\left< \KPC \right> + B\left< \KPD \right>.
\end{align*}

The standard unknot (with no crossing) is associated with the value $\delta$. Our desire to make $\left< \cdot \right>$ invariant under $\mathcal{R}^2$ and $\mathcal{R}^3$ forces $B = A^{-1}$.

For an oriented link diagram, its crossing has a sign. A positive crossing $(+1)$ is the one associated with that crossing $L_+$ (in the skein relation above). Then a negative crossing $(-1)$ is the one associated with $L_-$ in that diagram. The writhe of the oriented link diagram, denoted by $w(D)$, is sum of the signs of all its crossings.

\begin{remark}
\label{cha:jones-polynomial-4}
  Adding or removing a kink ($\mathcal{R}^1$) changes the bracket by a factor of $-A^{\pm 3}$. This move also changes the writhe by $\pm 1$.
\end{remark}

\begin{theorem}
\label{cha:jones-polynomial-5}
The polynomial $p(D,A) = (-A)^{-3w(D)}\left< D \right>$ is an oriented link invariant.
\end{theorem}

By the above remark, the truth of the above theorem is easily established. Indeed one gets $V_L(t)$ from $p_D(A)$ by putting $A = t^{-1/4}$. This model of the Jones polynomial using the bracket has some advantages which we will utilize in the pages that follow. 

\begin{remark}
\label{cha:jones-polynomial-6}
  Since $V_L(t)$ has only powers of $t^{\pm 1/2}$, it follows that $p(D,A)$ has only powers of $A^{\pm 2}$.
\end{remark}

Let $D$ be an oriented link diagram. Using the bracket skein relation, we resolve all the crossings one by one. Each crossing can be resolved in two ways, one that gets multiplied by $A$ in the skein relation (we call this $A$-smoothing or $A$-splitting) and other one that gets multiplied by $A^{-1}$ (we call this $B$-smoothing or $B$-splitting). A glance at the general skein relation reveals why we have chosen to call it $B$-smoothing rather than $A^{-1}$-smoothing.

A state $S$ is a map from the set of crossings of $D$ to the set $\{A, B\}$. If $D$ has $n$ crossings, then there are $2^n$ states. These are all the possible ways to resolve the crossings. Let $\mathcal{S}(D)$ denote the set of all states (later, we will turn $\mathcal{S}(D)$ into a category). For a state $S$, denote by $a(S)$ and $b(S)$, the number of $A$-smoothings and $B$-smoothings respectively in $S$. Put $\sigma(S) = a(S) - b(S)$. Once we resolve all the crossings, we end up with a bunch of topological circles. For a state $S$, we denote the number of circles by $\sharp S$.

\begin{theorem}
\label{cha:jones-polynomial-7}
  The bracket polynomial can be written as the following sum: 
\begin{equation}
\label{2:eq:2}
\left< D \right> = \sum_{S\in\mathcal{S}(D)}^{}  A^{\sigma(S)}(-A^2 - A^{-2})^{\sharp S - 1}.
\end{equation}
\end{theorem}

This theorem follows from the bracket skein relations. A state $S$ has $\sharp S$ circles, hence $\left< S \right> = (-A^2 - A^{-2})^{\sharp S - 1}$. Each $A$-smoothing gets multiplied by $A$ and each $B$-smoothing gets multiplied by $A^{-1}$. Thus each state $S$ contributes $A^{\sigma(S)}(-A^2 - A^{-2})$ to the sum where $\sigma = a(S) - b(S)$. This proves the theorem.

Let us call this the state-sum model of $\left< \cdot \right>$. We will make great use of it in proving propositions and theorems. See that the Jones polynomial $p(D,A)$ also admits a similar state-sum model. We list this result as a theorem due to its importance in the work that will follow.

\begin{theorem}
\label{cha:jones-polynomial-8}
The Jones polynomial $p(D, A)$ can be written as the following sum: 
\begin{equation}
\label{eq:1}
p(D,A) = (-1)^{-3w(D)} \sum_{S\in \mathcal{S}(D)}^{} A^{\sigma(S) - 3w(D)} (-A^2 - A^{-2})^{\sharp S - 1}.
\end{equation}
\end{theorem}

In our definition, we have normalized $\left< \KPA \right> = 1$. By applying the second skein relation, we must have $\left< \emptyset \right> = 1/(-A^2-A^{-2})$ for the empty link $\emptyset$. This is not a Laurent polynomial and does not look elegant. Thus we may normalize $\left< \KPA \right> = -A^2 - A^{-2}$ so that $\left< \emptyset \right> = 1$. We make no changes to the second and third skein relation. Thus the normalization preserves its invariance under $\mathcal{R}^2$ and $\mathcal{R}^3$. We will call this unreduced bracket polynomial and denote it by $\left< \hat{\cdot} \right>$. The same normalization applied to the Jones polynomial yields the unreduced Jones polynomial denoted by $\hat{p}(L,A)$.

One may normalize by giving $\left< \KPA \right>$ other values. One can introduce variable changes as well. For instance, substituting $A = t^{-1/4}$ in $p(L,A)$ yielded $V_L(t)$. Later when we study Khovanov homology we will substitute $q = -A^{-2}$ and work with $\hat{J}_L(q)$. Indeed one uses different normalizations and variable changes for different purposes and it may become difficult to know which one is which. For our work, we will use $\hat{J}_L(q)$ (introduced later) for Khovanov homology, $p(L,A)$ when we want to exploit the state-sum model (starting next chapter) and $V_L(t)$ for other purposes.

For an oriented link $L$, one obtains its mirror-image, denoted by $\overline{L}$, by applying $\KPB \to \KPE$ or $\KPE \to \KPB$ (whichever applicable) to each of the crossings. A link is said to be \emph{amphicheiral} (or \emph{acheiral}) if it is ambient isotopic to its mirror image. A chiral link is one which is not amphicheiral. The simplest chiral knot is the trefoil since $3_1$ and $3_2$ are mirror images and $3_1 \ncong 3_2$. The Jones polynomial is useful in detecting chirality.

\begin{theorem}
\label{cha:jones-polynomial-9}
  Let $L$ be an oriented link and $\overline{L}$ its mirror-image. Then 
\begin{equation}
\label{eq:2}
V_{\overline{L}}(t) = V_L(t^{-1}), \ p(\overline{D}, A) = p(D, A^{-1}).
\end{equation}
\end{theorem}

\begin{proof}
  Since the two equations are equivalent, we will establish only $p(\overline{D}, A) = p(D, A^{-1})$. In taking mirror-image, we apply $\KPB \to \KPE$ or $\KPE \to \KPB$. This reverses the sign of all the crossings. Thus $w(\overline{D}) = -w(D)$. From the skein relation $\left< \KPB \right> = A\left< \KPC \right> + A^{-1}\left< \KPD \right>$, one sees that the bracket $\left< \overline{D} \right>$ is just $\left< D \right>$ with $A$ and $A^{-1}$ exchanged. By Theorem \ref{cha:jones-polynomial-5}, it follows that $p(\overline{D}, A) = p(D, A^{-1})$.
\end{proof}

\begin{corollary}
\label{cha:jones-polynomial-10}
  Let $K$ be a knot. If the Jones polynomial $V_K(t)$ is not symmetrical under the exchange $t \leftrightarrow t^{-1}$ or equivalently, $V_K(t) \ne V_K(t^{-1})$, then $K$ is chiral.
\end{corollary}

The truth is obvious since $V_K(t)$ is an invariant. Thus we have found the first major use of $V_L(t)$. One may quickly verify that $V_{3_1}(t) \ne V_{3_1}(t^{-1})$.

\begin{remark}
\label{cha:jones-polynomial-11}
The converse of the above corollary is not true and hence this method cannot detect all the chiral knots. There exist chiral knots $K$ with $V_K(t) = V_K(t^{-1})$.
\end{remark}

Note that not all polynomial invariants can detect chirality. The Alexander polynomial cannot detect chirality.

\begin{remark}
\label{cha:jones-polynomial-12}
In Theorem \ref{cha:jones-polynomial-9}, we considered oriented link $L$ while in Corollary \ref{cha:jones-polynomial-10}, we considered knot $K$. The reason will become evident from the next few results. 
\end{remark}

\begin{proposition}
  \label{cha:jones-polynomial-13}
  The Jones polynomial $V_K(t)$ of a knot $K$ is independent of orientation.
\end{proposition}

\begin{proof}
\label{cha:jones-polynomial-14}
By the Jones skein relation ($\ref{2:eq:1}$), reversing the orientation of a knot leaves $L_+$ and $L_-$ unchanged (signs of crossings are preserved). The result follows.
\end{proof}

\begin{remark}
\label{cha:jones-polynomial-15}
By the same logic, reversing the orientation of all the components of a link $L$ also leaves $V_L(t)$ unchanged.
\end{remark}

In general, changing orientation of some (but not all) components of a link $L$ may change $V_L(t)$. A simple example is the Hopf link. In the following is a brilliant result that says that changing orientation of some components of a link multiplies $V_L(t)$ by some power of $t$. Before that we need a definition. Let $L$ be an oriented $n$-component link with components labelled $L^i$ for $1 \leq i \leq n$. The linking number $lk(L^i, L^j)$ is half of the sum of signs of crossings where a segment of $L^i$ crosses $L^j$. These signs ($+1$ or $-1$) follow the same convention as the writhe. Given an oriented link diagram with $n$ components, there are such $\binom{n}{2}$ linking numbers corresponding to each pair of components.

\begin{proposition}
\label{cha:jones-polynomial-16}
The linking number is an invariant.
\end{proposition}

\begin{proof}
The move $\mathcal{R}^1$ affects only one component of a diagram and can, thus, be ruled out. The move $\mathcal{R}^2$ introduces or deletes two crossings with opposite sign which does not affect $lk(L^i, L^j)$. Similarly $\mathcal{R}^3$ also leaves $lk(L^i, L^j)$ unchanged.
\end{proof}

For a link $L$, we write $L = \coprod^n L^i$ to specify that $L$ is an $n$-component link with each component labelled $L^i$. We may write simply $L = \coprod L^i$ if $n$ is unimportant or unknown. For greater generality, we extend this notation and write $L = M\coprod N$ where $M$ and $N$ are possibly links themselves. We write $rN$ to denote the same link $N$ but orientation of each component reversed.

We extend the definition of linking number as well. For oriented links $M$ and $N$, define $lk(M,N) = \sum_{i,j}^{} lk(M^i, N^j)$, where the sum is over all possible pairs of components of $M$ and $N$.

\begin{theorem}
\label{cha:jones-polynomial-17}
Let $L = M \coprod N$ be an oriented link. Let $L^{\prime} = M \coprod rN$ be the link obtained by reversing the orientation of all the components of $N$. Let $\lambda = lk(M,N)$. Then
\begin{equation}
\label{eq:3}
V_{L^{\prime}}(t) = t^{-3\lambda}V_L(t).
\end{equation}
\end{theorem}

\begin{proof}
  Recall that $V_L(t) = p(L,A) = (-A)^{-3w(L)}\left< L \right>$ and $\left< \cdot \right>$ is invariant under $\mathcal{R}^2$ and $\mathcal{R}^3$. Clearly $\left< L \right>$ is unaffected by change in orientation in any number of components of $L$. Thus only the writhe is changed. See that changes in crossing sign happen only at the sites where some $M^i$ meets some $N^i$. Observe that $lk(M,N)$ involves the same crossings. We denote a crossing between $M^i$ and $N^i$ by $x(M^i,N^j)$. We denote its sign by sgn$(x(M^i,N^j))$. Since writhe decreases by $2$ when a positive crossing changes to a negative crossing and increases by $2$ when a negative crossing changes to a positive, it follows that 
\begin{equation}
w(L^{\prime}) = w(L) - 2 \sum_{i,j,x}^{}\text{sgn} (x(M_i,N_j)) = w(L) - 4lk(M,N).
\end{equation}
From the above equation, it follows that 
\begin{equation}
\label{eq:4}
(-A)^{-3w(L)} = (-A)^{-3w(L^{\prime})}(-A)^{-12\cdot lk(M,N)}.
\end{equation}
Substituting $A^{-4} = t$ and $\lambda = lk(M,N)$, we have $V_{L^{\prime}}(t) = t^{-3\lambda} V_L(t)$.
\end{proof}

Notwithstanding that $V_L(t)$ depends on the orientation of some of the components of $L$, Theorem \ref{cha:jones-polynomial-17} equips us with a tool to control this dependence. Indeed, one may define a variant of the Jones polynomial which does not depend on the orientation. The construction is obvious and straightforward.

Define self-writhe of a link $L$ as $\hat{w}(L) = w(L) - 2 \sum_{i,j}^{}lk(L^i,L^j)$. Alternatively, $\hat{w}(L) = \sum_i^{}w(K^i)$, where $K^i$ is $L^i$ considered as a knot (that is, ignoring every other component $L^j$ where $i \ne j$). Obviously $\hat{w}(L)$ is $\mathcal{R}^{2,3}$-invariant and gets multiplied by $A^{\pm 1}$ under $\mathcal{R}^1$ in the way $w(L)$. Clearly $\hat{w}(D)$ is independent of orientation (by the same argument used in the proof of Proposition \ref{cha:jones-polynomial-13}). Define $f(L, A) = (-A)^{-3\hat{w}(L)}\left< L \right>$ and then put $A = t^{-1/4}$ to define $V^{\prime}_L(t)$. The Jones polynomial $V^{\prime}_L(t)$ is then defined and is an invariant for unoriented links.

There are two two-variable polynomial invariants that generalize $V_L(t)$. They are the HOMFLY (or HOMFLYPT) polynomial and the Kauffman polynomial. The HOMFLY polynomial is defined (indeed, well-defined) by the following theorem.

\begin{theodef}
\label{cha:jones-polynomial-19}
There exists a unique map $P$ from $\mathcal{L}/(\cong)$, the set of oriented links under equivalence of ambient isotopy, to $\Z[l^{\pm}, m^{\pm}]$ such that $P(\KPA) = 0$ and 
\begin{equation}
\label{eq:5}
lP(L_+) + l^{-1}P(L_-) + mP(L_0) = 0,
\end{equation}
where $L_+, L_-$ and $L_0$ differ only locally at the points indicated below.
\begin{figure}[h]
  \centering
  \includegraphics[scale=.04]{images/jones-skein-relation.png}
\end{figure}
\end{theodef}
The polynomial $P$ is called the HOMFLY polynomial. The proof for existence and uniqueness relies on induction on the number of crossings \cite{lickorish1997}. The HOMPLY polynomial may well be regarded as polynomial in three variables $x,y,z$ hence belonging to the ring $\Z[x^{\pm}, y^{\pm}, z^{\pm}]$ with the skein relation $xP(L_+) + yP(L_-) + zP(L_0) = 0$.

\begin{remark}
The condition $P(\KPA) = 0$ applies to all knots isotopic to $\KPA$. Indeed, this is already implied in the definition since the domain of $P$ is $\mathcal{L}/(\cong)$.
\end{remark}

\begin{theorem}
\label{cha:jones-polynomial-18}
There exists a unique map $\Theta$ from the set of unoriented links to $\Z[a^{\pm}, z^{\pm}]$ such that 
\begin{enumerate}[font=\upshape]
\item\label{item:1} Normalization: $\Theta(\KPA) = 0$ (the crossing-less unknot);
\item\label{item:2} Invariance: $\Theta(\cdot)$ is $\mathcal{R}^{2,3}$-invariant; 
\item\label{item:3} Effect under $\mathcal{R}^1$: $\Theta(s_r) = a\Theta(s)$, $\Theta(s_l) = a^{-1}\Theta(s)$; 
\item\label{item:4} Skein relation: 
\begin{equation}
\label{3:eq:6}
\Theta(\KPB) + \Theta(\KPE) = z\Theta(\KPC) + z\Theta(\KPD),
\end{equation}
\end{enumerate}
where $s$ is an arc and $s_r$ (resp. $s_l$) is the same arc with a right-handed (resp. left-handed) curl added ($\mathcal{R}^1$).
\end{theorem}

\begin{definition}
\label{cha:jones-polynomial-21}
The Kauffman polynomial is a map from the set of unoriented links to $\Z[a^{\pm}, z^{\pm}]$ defined by $K(L) = a^{-w(L)} \Theta (L)$, where $\hat{w}(L)$ is the self-writhe of $L$.
\end{definition}

\begin{remark}
The writhe $w(D)$ in the definition may be replaced by $\hat{w}(D)$ since both are $\mathcal{R}^{2,3}$-invariant and $a^{-w(L)} \Theta (L)$ is an invariant for unoriented links as well. Indeed, using $\hat{w}(L)$ would have been a better choice since it does not require computation of $lk(L^i,L^j)$ but the use of $w(L)$ is standard and well-established.
\end{remark}

The new polynomials $P(L)$ and $K(L)$ share several of the properties we derived for $V_L(t)$.

\begin{remark}
  As we have seen in this chapter and will see in the subsequent chapters, some problems are easier solved using $p(L,A)$ while others are easier with $V_L(t)$.
\end{remark}


% mutation effects