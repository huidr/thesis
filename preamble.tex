\pagestyle{plain}
%\setlength{\abovedisplayskip}{0pt}
%\setlength{\belowdisplayskip}{0pt}

% space before and after equations
\expandafter\def\expandafter\normalsize\expandafter{%
    \normalsize%
    \setlength\abovedisplayskip{8pt}%
    \setlength\belowdisplayskip{8pt}%
    \setlength\abovedisplayshortskip{4pt}%
    \setlength\belowdisplayshortskip{4pt}%
}

%\setlength{\parskip}{0pt}
%\setlength{\parindent}{0pt}
\definecolor{myColor}{RGB}{193,225,193}
\newcommand{\B}{\ensuremath{\mathcal{B}}}
\newcommand{\A}{\ensuremath{\mathcal{A}}}
\linespread{1.3}

\patchcmd{\tocchapter}{#1}{\MakeUppercase{#1}}{}{}
\setcounter{tocdepth}{0}

%\renewcommand\@pnumwidth{1em} % <-- depending on the total number of pages
%\patchcmd{\@tocline}
%  {\hfil}
%  {\leaders\hbox{\,.\,}\hfil}
%  {}{}
%\def\l@subsection{\@tocline{2}{0pt}{3.4pc}{5pc}{}}
%\makeatother

%\DeclareRobustCommand{\gobblefour}[5]{}
%\newcommand*{\SkipTocEntry}{\addtocontents{toc}{\gobblefour}}

%\renewcommand{\thesection}{\thechapter.\arabic{section}}
%\renewcommand{\baselinestretch}{1.1}\normalsize

\usetikzlibrary{cd}
\usetikzlibrary{babel}
\usetikzlibrary{knots}

%%%%%%%%%%%%%%%%%%%%%%%%%%%%%%%%%%%%%%%%%%%%%%%%%%%%%%%%%%%%%%%%%%%%%%%%%%%
%%%%% SKEIN RELATIONS
%%%%%%%%%%%%%%%%%%%%%%%%%%%%%%%%%%%%%%%%%%%%%%%%%%%%%%%%%%%%%%%%%%%%%%%%%%%

\makeatletter

% We define separate versions (large and small) for the frames:
\newcommand*\@KP@Large@frame[2]{%
    \setlength\unitlength{\fontdimen 22 #1\tw@}%
    \vrule \@width\z@ \@height 4\unitlength \@depth\tw@\unitlength
    \begin{picture}(6,2)(-3,-1)%
        \def\@KP@Radius     {3}%
        \def\@KP@Hole@radius{.5}% The same value seem adequate for both...
        \def\@KP@Diameter   {6}%
        #2%
    \end{picture}%
}
\newcommand*\@KP@Small@frame[2]{%
    \setlength\unitlength{\fontdimen 22 #1\tw@}%
    \vrule \@width\z@ \@height \thr@@\unitlength \@depth\@ne\unitlength
    \begin{picture}(4,2)(-2,-1)%
        \def\@KP@Radius     {2}%
        \def\@KP@Hole@radius{.5}% ... but let it be customizable too.
        \def\@KP@Diameter   {4}%
        #2%
    \end{picture}%
}

% On the other hand, for the commands that draw the four different shapes, it
% seems that all differences between the small variant and the large one can be
% confined in the following three macros (here, we just declare their name):
\newcommand*\@KP@Radius     {}
\newcommand*\@KP@Hole@radius{}
\newcommand*\@KP@Diameter   {}
%
% The four shapes:
\newcommand*\@KP@Shape@A{%
    \put(0,0){\circle{\@KP@Diameter}}%
}
\newcommand*\@KP@Shape@B{%
    \Line(-\@KP@Radius,\@KP@Radius )(\@KP@Radius,-\@KP@Radius)%
    \Line(-\@KP@Radius,-\@KP@Radius)(-\@KP@Hole@radius,-\@KP@Hole@radius)%
    \Line(\@KP@Radius ,\@KP@Radius )(\@KP@Hole@radius ,\@KP@Hole@radius )%
}
\newcommand*\@KP@Shape@C{%
    \cbezier(-\@KP@Radius,\@KP@Radius )(0,0)(0,0)(\@KP@Radius,\@KP@Radius )%
    \cbezier(-\@KP@Radius,-\@KP@Radius)(0,0)(0,0)(\@KP@Radius,-\@KP@Radius)%
}
\newcommand*\@KP@Shape@D{%
    \cbezier(-\@KP@Radius,-\@KP@Radius)(0,0)(0,0)(-\@KP@Radius,\@KP@Radius)%
    \cbezier(\@KP@Radius ,-\@KP@Radius)(0,0)(0,0)(\@KP@Radius ,\@KP@Radius)%
}
\newcommand*\@KP@Shape@E{%
    \Line(-\@KP@Radius,-\@KP@Radius)(\@KP@Radius,\@KP@Radius)%
    \Line(-\@KP@Hole@radius,\@KP@Hole@radius)(-\@KP@Radius,\@KP@Radius)%
    \Line(\@KP@Radius,-\@KP@Radius)(\@KP@Hole@radius,-\@KP@Hole@radius)%
}
\newcommand*\@KP@Shape@F{%
    \Line(-\@KP@Radius,\@KP@Radius )(\@KP@Radius,-\@KP@Radius)%
    \Line(-\@KP@Radius,-\@KP@Radius)(\@KP@Radius,\@KP@Radius)%
}  
\newcommand*\@KP@Shape@Z{%
    \Line(-\@KP@Radius,-\@KP@Radius)(\@KP@Radius,\@KP@Radius)%
    \Line(-\@KP@Hole@radius,\@KP@Hole@radius)(-\@KP@Radius,\@KP@Radius)%
    \Line(\@KP@Radius,-\@KP@Radius)(-\@KP@Hole@radius,\@KP@Hole@radius)%
}
\newcommand*\@KP@Atomic@mathpalette[1]{%
    \mathinner{% or "\mathord"?
        % Note that a new level of grouping has just been entered (p. 290).
        % \color{gray}% not used, for now
        \mathchoice{%
            \linethickness{.6\p@}% Tip: use thicker lines if you decide to 
                                 % revert to using gray.
            \@KP@Large@frame \textfont {#1}%
        }{%
            \linethickness{.4\p@}% adjustable
            \@KP@Small@frame \textfont {#1}%
        }{%
            \linethickness{.3\p@}% adjustable
            \@KP@Small@frame \scriptfont {#1}%
        }{%
            \linethickness{.2\p@}% adjustable
            \@KP@Small@frame \scriptscriptfont {#1}%
        }%
    }%
}

% User-level commands:
\newcommand*\KPA{\@KP@Atomic@mathpalette \@KP@Shape@A}
\newcommand*\KPB{\@KP@Atomic@mathpalette \@KP@Shape@B}
\newcommand*\KPC{\@KP@Atomic@mathpalette \@KP@Shape@C}
\newcommand*\KPD{\@KP@Atomic@mathpalette \@KP@Shape@D}
\newcommand*\KPE{\@KP@Atomic@mathpalette \@KP@Shape@E}
\newcommand*\KPF{\@KP@Atomic@mathpalette \@KP@Shape@F}
\newcommand*\KPZ{\@KP@Atomic@mathpalette \@KP@Shape@Z}

\makeatother

%%%%%%%%%%%%%%%%%%%%%%%%%%%%%%%%%%%%%%%%%%%%%%%%%%%%%%%%%%%%%%%%%%%%%%%
%%%%%%%%%%%%% OTHER KNOTS, TANGLES
%%%%%%%%%%%%%%%%%%%%%%%%%%%%%%%%%%%%%%%%%%%%%%%%%%%%%%%%%%%%%%%%%%%%%%%

\newcommand*\PKINK{
\begin{tikzpicture}
\begin{knot}[
  consider self intersections,
  flip crossing = 1,
]
\strand (0,0) .. controls +(3,1) and +(-3,1) .. (1,0);
\end{knot}
\end{tikzpicture}
}

\newcommand*\NKINK{
\begin{tikzpicture}
\begin{knot}[
  consider self intersections,
]
\strand (0,0) .. controls +(3,1) and +(-3,1) .. (1,0);
\end{knot}
\end{tikzpicture}
}

\newcommand*\ARC{
\begin{tikzpicture}
\begin{knot}[]
\strand (0,0) .. controls +(3,1) and +(-3,1) .. (1,0);
\end{knot}
\end{tikzpicture}
}

\geometry{
 a4paper,
 left=1.25in,
 right=1in,
 top=1in,
 bottom=1in
}