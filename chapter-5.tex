\chapter{Categorification}
\label{chapter5}

\section{Homology theories}

The Euler characteristic $\chi$ is a topological invariant. That is, two topological spaces belonging to the same homeomorphism class have the same $\chi$. For a polyhedron, it is given by $\chi = v - e + f$, where $v, e$ and $f$ are respectively the number of vertices, edges and faces. One may generalize this to a definition valid for simplicial complexes. A simplicial complex is a set composed of simplexes. A $0$-simplex is just a point, $1$-simplex is an edge, $2$-simplex is a triangle (consisting of its interior), $3$-simplex is a solid tetrahedron and $n$-simplex is a higher-dimensional analogue. One easily sees that the definition of $\chi$ valid for simplicial complexes should then be 
\begin{equation}
\label{eq:1}
\chi(X) = k_0 - k_1 + k_2 - k_3 + \cdots = \sum_{n=0}^{\infty} (-1)^n k_n, 
\end{equation}
where $k_n$ is the number of $n$-simplexes in $X$. Putting $k_0 = v$, $k_1 = e$, $k_2 = f$ and $k_n = 0$ for higher $n$ reduces this definition to the polyhedron case. Still more generally, we may define free abelian groups $C_n(X)$ so that $\chi$ appears as an alternating sum of the ranks of these groups: 
\begin{equation}
\label{eq:2}
\chi(X) = \sum_n^{\infty} (-1)^n \text{ rk}C_n(X).
\end{equation}
One easily sees that $\text{rk } C_n(X) = k_n$ so that the definitions agree. Thus, $C_n$ must be free abelian groups generated by the $n$-simplexes.

The following chain of free abelian groups is commonly referred to as a chain complex.
\begin{equation}
\label{eq:3}
\cdots \to C^n \xrightarrow{\partial_n} C^{n-1} \xrightarrow{\partial_{n-1}} \cdots \xrightarrow{\partial_3} C_2 \xrightarrow{\partial_3} C_1 \xrightarrow{\partial_1} C_0 \to 0.
\end{equation}
In the above chain complex, all but finitely many $C_n$ are zero. Further, $\partial_n \partial_{n-1} = 0$ (commonly written as $\partial^2 = 0$), where the map $\partial$ (called the differential or boundary operator) maps simplexes to their boundaries. The condition $\partial^2 = 0$ (a necessary condition for chain complexes) is equivalent to saying that boundary of a boundary is zero (or rather, a boundary has no boundary). The homology groups $H_n(X)$ are defined as 
\begin{equation}
\label{eq:4}
H_n(X) = \frac{\ker \partial_n}{\text{im }\partial_{n-1}}.
\end{equation}

Further still, simplicial homology may be generalized to singular homology. We still have a similar looking chain complex with $\partial^2 = 0$ and the homology groups are defined in the same way. If the groups $H_n(X)$ have finite rank, it turns out that rk $(C_n) = $ rk $(H_n(X))$.

The homology groups $H_n(X)$ are more fundamental to the topological space $X$ than its Euler characteristic $\chi(X)$. Besides, $H_n$ is a functor from the category of chain complexes to the category of abelian groups.

Later, inspired by the homology theory for spaces, we will construct a homology theory for links such that the graded Euler characteristic of this homology turns out to be the Jones polynomial.

\section{Categories and functors}

Let $R$ be a commutative ring. Chain complexes of $R$-modules form a category $Kom_R$ (sometimes denoted by $Ch_R$ in some texts) in which objects are chain complexes $(C_{*},\partial_{*})$ and a morphism from a chain complex $(C_{*},\partial_{*})$ to another chain complex $(C^{\prime}_{*}, \partial^{\prime}_{*})$ is a sequence of homomorphisms $\psi_n : C_n \to C^{\prime}_n$ such that $\psi_{n-1} \circ \partial_n = \partial^{\prime}_n \circ \psi_n$ for all $n$. That is, the following diagram
\begin{equation}
\label{eq:6}
\begin{tikzcd} 
  \cdots \arrow[r, "\partial_{n+2}"] & C_{n+1} \arrow[r, "\partial_{n+1}"] \arrow[d, "\psi_{n+1}"] & C_n \arrow[r, "\partial_n"] \arrow[d, "\psi_n"] & C_{n-1} \arrow[d, "\psi_{n-1}"] \arrow[r, "\partial_{n-1}"] & \cdots\\
  \cdots \arrow[r, "\partial^{\prime}_{n+2}"] & C^{\prime}_{n+1} \arrow[r, "\partial^{\prime}_{n+1}"] &  C^{\prime}_{n} \arrow[r, "\partial^{\prime}_n"] & C^{\prime}_{n-1} \arrow[r, "\partial^{\prime}_{n-1}"] & \cdots
\end{tikzcd}
\end{equation}
commutes. As noted above, $H_n$ is a functor from $Kom_R$ to $R$-\textbf{Mod} (abelian groups are $\Z$-modules).

Let $X, Y$ be spaces and $f: X \to Y$ be a continuous map (hence a morphism in \textbf{Top}). Then $f$ induces a chain map or morphism $(C_{*}(X), \partial_{*}(X)) \to (C_{*}(Y), \partial_{*}(Y))$ in $Kom_R$. From the discussion above, it follows that $H_n$ is a functor from \textbf{Top} to $R$-\textbf{Mod}.

% \section{Categorification of the Jones polynomial}
