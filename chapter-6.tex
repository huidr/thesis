\chapter{Khovanov homology}
\label{chapter6}

\section{Categorification of $\hat{J}_L(q)$}

The bracket polynomial may be normalized in the following way: 
\begin{align*}
  \left[ \KPA \right] &= q+q^{-1}, \\
  \left[ \KPA D \right] &= (q+q^{-1}) \left[ D \right], \\
  \left[ \KPB \right] &= \left[ \KPC \right] - q^{-1}\left[ \KPD \right].
\end{align*}

Observe that we used the symbol $[D]$ to denote the bracket polynomial defined in this way.
\begin{proposition}
\label{sec:cohom-groups-mathc-1}
The bracket $[D]$ is $\mathcal{R}^{2,3}$ but changes by a factor of $q^{\pm 1}$ or $-q^{\pm 2}$ under $\mathcal{R}^1$.
\end{proposition}

This is easily checked.

Let $n$ be the number of crossings in the link diagram $D$. Let $n_+$ and $n_-$ be the number of positive crossings and negative crossings in $D$ such that $w(D) = n_+ - n_-$. Define the unnormalized Jones polynomial as
\begin{equation}
\label{eq:3}
\hat{J}(D) = (-1)^{n_-} q^{n_+ - 2n_-} [D].
\end{equation}

\begin{theorem}
\label{sec:cohom-groups-mathc-4}
The polynomial $\hat{J}(D)$ is an invariant for oriented links. The usual Jones polynomial $J(D)$ is related to $\hat{J}(D)$ by the relation 
\begin{equation}
\label{eq:4}
J(D) = \hat{J}(D)/(q + q^{-1}).
\end{equation}
\end{theorem}

\begin{proof}
\label{sec:cohom-groups-mathc-5}
The term $(-1)^{n_-} q^{n_+ - 2n_-}$ is $\mathcal{R}^{2,3}$-invariant. By Proposition \ref{sec:cohom-groups-mathc-1}, adding or removing a positive kink multiplies the bracket by a factor of $q^{\pm 1}$ and adding or removing a negative kink multiplies the bracket by a factor of $-q^{\pm 2}$. The term $(-1)^{n_-} q^{n_+ - 2n_-}$ negates the changes in bracket polynomial under $\mathcal{R}^1$. Since $\hat{J}(D)$ is normalized by giving the unknot a value of $q + q^{-1}$, we simply divide $\hat{J}(D)$ by this value to get the usual Jones polynomial $J(D)$.
\end{proof}
\begin{remark}
\label{sec:cohom-groups-mathc-7}
One may obtain the Jones polynomial $V_L(t)$ by substituting $q = -t^{1/2}$.
\end{remark}

Using the skein relation, each crossing of $D$ can be smoothed in two ways---$1$-smoothing (in which bracket polynomial gets multiplied by $q^{-1}$) and $0$-smoothing. Let $\sharp 1$ be the number of $1$-smoothings and $\sharp S$ the number of circles (loops) for a state $S$.

\begin{proposition}
\label{sec:cohom-groups-mathc-6}
Let $\mathcal{S}$ be the set of all states of an oriented link diagram $D$. Then the bracket $[D]$ can be written as a state-sum: 
\begin{equation}
\label{eq:5}
[D] = \sum_{S \in \mathcal{S}}^{} (-1)^{\sharp 1} q^{\sharp 1} (q + q^{-1})^{\sharp S}.
\end{equation}
\end{proposition}

\begin{proof}
\label{sec:cohom-groups-mathc-8}
Each state $S$ has $\sharp S$ circles and each circle has the bracket $q + q^{-1}$. Each time a crossing receives a $1$-smoothing, the bracket changes by a multiple of $-q$. Since state $S$ has $\sharp 1$ number of $1$-smoothings, it is easy to see that each contributes $(-1)^{\sharp 1} q^{\sharp 1} (q + q^{-1})^{\sharp S}$ to the sum. 
\end{proof}

\begin{theorem}
\label{sec:cohom-groups-mathc-9}
The unnormalized Jones polynomial $\hat{J}(D)$ can be written as a state-sum: 
\begin{equation}
\label{eq:6}
\hat{J}(D) = \sum_{S\in \mathcal{S}}^{} (-1)^{n_- + \sharp 1} q^{n_+ - 2n_- + \sharp 1} (q+q^{-1})^{\sharp S}.
\end{equation}
\end{theorem}

\begin{proof}
\label{sec:cohom-groups-mathc-10}
It follows trivially from the definition of $\hat{J}(D)$ and Proposition \ref{sec:cohom-groups-mathc-6}.
\end{proof}

Assign an arbitrary order of the crossings of $D$. A state $S$ may be represented by a binary sequence of length $n$. The bit $0$ or $1$ denote the type of smoothing for each crossing. That is, any state is of the form $x_1x_2\cdots x_n$ where $x_i = 0$ if the $i$th crossing has $0$-smoothing, otherwise $x_i = 1$. See that there is a one-to-one correspondence between the set $\mathcal{S}$ of states and the set of all binary sequences of length $n$.

The states in $\mathcal{S}$ form a category, denoted by $\mathcal{S}(D)$. The objects in $\mathcal{S}(D)$ are the states of $D$ (represented as binary sequences) while morphisms $S_1 \to S_2$ are such that $S_2$ is obtained from $S_1$ by changing a single $0$-smoothing in $S_1$ to $1$-smoothing. Speaking in terms of binary sequences, we have the morphism 
\begin{displaymath}
x_1\cdots x_i \cdots x_n \to x_1\cdots x_i^{\prime} \cdots x_n,
\end{displaymath}
where $x_i = 0$ and $x_i^{\prime} = 1$, and every other bit is left unchanged.

There is a notation for such a morphism. We use the asterisk $*$ to denote where the change of smoothing from $0$ to $1$ occurs. For instance,
\begin{displaymath}
000 \xrightarrow{0*0} 010 \xrightarrow{*10} 110 \xrightarrow{11*} 111.
\end{displaymath}
The state category $\mathcal{S}(D)$ can be given some extra structure. Each state object has a height $\sharp 1$ (the number of $1$-smoothings) associated with it. For instance, $0\cdots0$ is the only state with height $0$ and $1\cdots 1$ is the only state with height $n$. Every other state has a height between $0$ and $n$. All state objects can then be represented in a diagram according to their heights. All state objects with the same height are placed in the same column and the arrows go from left to right in the direction of increasing height. The following are state categories for the Hopf link and the trefoil.

\begin{figure}[h]
  \centering
  \includegraphics[scale=.35]{images/Hopfcat.png}
\end{figure}

\begin{figure}[h]
  \centering
  \includegraphics[scale=.35]{images/cubecat.png}
\end{figure}

A state diagram is simply a bunch of circles. Let $S_1 \to S_2$ be a morphism in the state category $\mathcal{S}(D)$. Now we see $S_i$ as a bunch of circles and not as a binary sequence. Observe that two things can happen here. Either two circles in $S_1$ fuse together to form a circle in $S_2$ or a circle in $S_1$ bifurcates into two circles $S_2$, while every other circle remain untouched. Observe that in each case $S_1 \to S_2$ can be thought of as a surface cobordism since a disjoint union of circles form a $1$-manifold. These cobordisms are directed down the page as shown below.

\begin{figure}[h]
  \centering
  \includegraphics[scale=.1]{images/cobordism.png}
\end{figure}

From the discussion above, we have another category whose objects are states seen as disjoint union of circles and whose morphisms are surface cobordisms. It is taken care that this morphism (surface cobordism) is one of the two types shown above. We denote this category by $Cob\mathcal{S}(D)$. See that $Cob\mathcal{S}(D)$ shares the same height structure we associated to $\mathcal{S}(D)$ although this has become less apparent. This can be seen from the equivalence of the categories $Cob\mathcal{S}(D)$ and $\mathcal{S}(D)$. An appropriate functor $\mathcal{S}(D) \to Cob\mathcal{S}(D)$ that builds this equivalence should assign to each object of $Cob\mathcal{S}(D)$ its correct height.

\begin{proposition}
\label{sec:cohom-groups-mathc-11}
Let $\mathcal{A}$ be a category. Let $Mat(\mathcal{A})$ be the set of all possible tuples of objects of $\mathcal{A}$. A morphism from a $p$-tuple $\mathcal{O} = (\mathcal{O}_1, \ldots, \mathcal{O}_p)$ to a $q$-tuple $\mathcal{O}^{\prime} = (\mathcal{O}_1^{\prime}, \ldots, \mathcal{O}_q^{\prime})$ is defined to be the matrix $[d_{ij}]_{p\times q}$ where $d_{ij}$ is the morphism $\mathcal{O}_i \to \mathcal{O}_j^{\prime}$. Then $Mat(\mathcal{A})$ forms a category in which objects are tuples of objects of $\mathcal{A}$ and morphisms are the matrices as defined above. The composition of morphisms is given by matrix multiplication.
\end{proposition}

The proof follows by construction. One simply checks that all requirements of a category are satisfied.

Look at $Mat(Cob\mathcal{S}(D))$. The category $Mat(Cob\mathcal{S}(D))$ is constructed from $Cob\mathcal{S}(D)$ in the way $Mat(\mathcal{A})$ is constructed from $\mathcal{A}$ in Proposition \ref{sec:cohom-groups-mathc-11}. Further, we ensure that the following two conditions are satisfied. 
\begin{enumerate}
\item\label{item:1} If there is no morphism (cobordism) $S_i \to S_j$ in $Cob\mathcal{S}(D)$, we equate $d_{ij} = 0$.
\item\label{item:2} Each object in $Mat(Cob\mathcal{S}(D))$ is of the form $\mathcal{O}^h = (S_1, \ldots, S_k)$ where each $S_i$ has height $h$ (that is, $\sharp 1 = h$) and these are all the states with height $h$.
\end{enumerate}

These two conditions ensure that we end up with something like the following: 
\begin{equation}
\label{eq:7}
\mathcal{O}^{0} \xrightarrow{d^0} \mathcal{O}^1 \xrightarrow{d^1} \mathcal{O}^2 \xrightarrow{d^2} \cdots \xrightarrow{d^{n-1}} \mathcal{O}^{n}.
\end{equation}

In the above, we have collapsed all states with the same number of $1$-smoothings into an object $\mathcal{O}^h$. For the sake of completion, we may add the initial object $\hat{0}$ and the final object $\hat{1}$ in $Mat(Cob\mathcal{S}(D))$. In fact, we can equate $\hat{0} = \hat{1}$ if we define both of them to be empty tuples. Then by this construction $Mat(Cob\mathcal{S}(D))$ becomes a pointed category (a pointed category is one with a zero element, that is, the initial object and final object exist and are isomorphic).
\begin{equation}
\label{eq:8}
\hat{0} \xrightarrow{0} \mathcal{O}^{0} \xrightarrow{d^0} \mathcal{O}^1 \xrightarrow{d^1} \mathcal{O}^2 \xrightarrow{d^2} \cdots \xrightarrow{d^{n-1}} \mathcal{O}^{n} \xrightarrow{0} \hat{0}.
\end{equation}

The morphism $0$ in the above category is zero map, that is, it satisfies $d^i \cdot 0 = 0 = 0\cdot d^i$.

This reminds us of a chain complex except that $\mathcal{O}^h$ are tuples of states and not abelian groups (or modules) and we have not established that $d^{i+1} \circ d^{i} = 0$. But both of these problems can be fixed and we can indeed have a chain complex this way. We shall come back to this later. Now we digress to something else.

A graded vector space is a decomposition of a vector space into a direct sum of subspaces. Let $W$ be a $\Z$-graded vector space. Then 
\begin{displaymath}
W = \bigoplus_{m\in\Z} W_m,
\end{displaymath}
where each $W_m$ is a subspace of $W$. The subspace $W_m$ is said to have degree $m$ and its elements are said to be homogeneous elements of degree $m$.

Although $\Z$ grading is used above, any indexing set $\mathcal{I}$ may also be used. We stick to $\Z$ grading. The graded dimension of $W$, denoted by $\dim_q W$, is defined to be the power series
\begin{equation}
\label{eq:9}
\dim_q W(q) \coloneqq \sum_{m \in \Z}^{} q^m \dim(W_m).
\end{equation}

\begin{proposition}
\label{sec:cohom-groups-mathc-12}
Let $W = \bigoplus_m W_m$ and $W^{\prime} = \bigoplus_m W_m^{\prime}$ be two $\Z$-graded vector spaces. Then 
\begin{align}
\label{eq:1}
  \dim_q (W \oplus W^{\prime}) &= \dim_q W + \dim_q W^{\prime}, \\
\label{eq:2}
  \dim_q (W \otimes W^{\prime}) &= \dim_q W \cdot \dim_q W^{\prime}.
\end{align}
\end{proposition}

The proof follows from the definition given by Equation \ref{eq:9} and expansion of the terms. 

\begin{proposition}
\label{sec:cohom-groups-mathc-13}
Given a $\Z$-graded vector space $W = \bigoplus_{m} W_m$, define shifting operation, denoted by $\{l\}$, by $W\{l\}_m \coloneqq W^{m-l}$ so that $W\{l\} = \bigoplus_m W\{l\}_m$. Then 
\begin{equation}
\label{eq:10}
\dim_qW\{l\} = q^l \dim_q W.
\end{equation}
\end{proposition}

Let $V = \F_2[x]/(x^2)$. We may also use $\Z$ or $\Q$ instead of $\F_2$. This algebraic structure can be viewed in different ways. First, $V$ can be realized as a graded vector space. A basis would be $\{1, x\}$. By grading  $\deg(1) = 1$ and $\deg(x) = -1$, it follows that $\dim_q V = q + q^{-1}$. This choice of grading is not random since the bracket polynomial has been normalized so that its value for the crossingless unknot is $q+q^{-1}$.

Consider the $k$th tensor product $V^{\otimes k}$. Then $\dim_q V^{\otimes k} = (q + q^{-1})^k$. By shifting $\{r\}$, it follows that $\dim_q V^{\otimes k}\{r\} = q^r (q + q^{-1})^k$. Substituting $r = n_+ - 2n_- + \sharp 1$ and $k = \sharp S$, it follows that 
\begin{equation}
\label{eq:11}
\dim_q V^{\otimes \sharp S} \{ n_+ - 2n_- + \sharp 1 \} = q^{n_+ - 2n_- + \sharp 1} (q + q^{-1})^{\sharp S}.
\end{equation}
Comparing this form with the state-sum model of the unnormalized Jones polynomial given by Equation \ref{eq:6}, we see an almost perfect resemblance. Indeed, this has been our intention all along. But before hitting the jackpot, we have to digress yet again.

As mentioned above, $V$ may be viewed in different ways. Endowed with some required operations, it becomes a Frobenius algebra. Recall that a Frobenius algebra is a finite-dimensional associative algebra with a bilinear form.
There is a nice duality theory associated with Frobenius algebras.

Let $m : V\otimes V \to V$ be a bilinear map defined as follows 
\begin{equation}
\label{eq:12}
1 \otimes 1 \rightsquigarrow 1, \hspace*{2em} 1 \otimes x \rightsquigarrow x, \hspace*{2em} x \otimes 1 \rightsquigarrow 1,  \hspace*{2em} x \otimes x  \rightsquigarrow 0.
\end{equation}
This operation is often called multiplication. Let $\Delta: V \to V \otimes V$ be defined as follows 
\begin{equation}
\label{eq:13}
x \rightsquigarrow x \otimes x, \hspace*{2em} 1 \rightsquigarrow 1 \otimes x + x \otimes 1.
\end{equation}

A monoidal (or tensor) category $\A$ is one which is equipped with an associative bifunctor $\otimes : \A \times \A \to \A$ and an object $I$ which is an identity (unit) for $\otimes$. The associativity and identity are up to a natural isomorphism.

Monoidal categories may be seen as a generalization of vector spaces, abelian categories and $R$-modules under tensor products. This way, \textbf{Ab}, the category of abelian groups and \textbf{Vect$_k$}, the category of vector spaces over a field $k$, are monoidal categories. Observe that $\Z$ is then the unit for \textbf{Ab} and the vector space of dimension $1$ is the unit for \textbf{Vect$_k$}. Both of these monoidal categories are special cases of \textbf{$R$-Mod}, the category of $R$-modules. A monoidal category is then a categorification of a monoid, whose elements are the isomorphism classes of the category's objects and whose binary operation is the tensor product.

The Frobenius algebra $V$ may be abstracted as a Frobenius object $(V, m, \eta, \Delta, \epsilon)$ in a monoidal category $(\A, \otimes, I)$, associated with four morphisms 
\begin{equation}
\label{eq:15}
m : V \otimes V \to V, \hspace{2em} \eta : I \to V, \hspace{2em} \Delta : V \to V \otimes V, \hspace{2em} \epsilon: V \to I
\end{equation}
such that $(V, m, \eta)$ is a monoid, $(V, \Delta, \epsilon)$ is a comonoid and the diagrams
\begin{equation}
\label{eq:14}
\begin{tikzcd} 
  V \otimes V \arrow[r, "\Delta \otimes V"] \arrow[d, "m"] & V \otimes V \otimes V \arrow[d, "V \otimes m"]\\
  V \arrow[r, "\Delta"] & V \otimes V
\end{tikzcd}
\hspace*{2em}
\begin{tikzcd}
  V \otimes V \arrow[r, "V \otimes \Delta"] \arrow[d, "m"] & V \otimes V \otimes V \arrow[d, "m \otimes V"]\\ 
  V \arrow[r, "\Delta"] & V \otimes V
\end{tikzcd}
\end{equation}

commute. A monoid is an object together with two morphisms: multiplication $m$ and unit $\eta$ satisfying some commutative diagrams. A comonoid is the dual of a monoid, has two morphisms: comultiplication $\Delta$ and counit $\epsilon$.

In our case, $1$ is a multiplicative unit, and $\F_2$ is the unit $I$ above. That is, $\eta: \F_2 \to V$ is given by $\eta(1_{\F_2}) = 1_V$. Dually, the counit $\epsilon: V \to \F_2$ is such that $\epsilon(1_V) = 1_{\F_2}$. Frobenius algebras are algebraic counterparts to $(1+1)$-dimensional topological quantum field theories (TQFTs).

In the simplest level, $(1+1)$-dimensional TQFTs are functors from \textbf{$2$-Cob}, the category of 2-dimensional cobordisms between $1$-dimensional manifolds, to \textbf{Vect$_k$}. The correspondence between Frobenius algebras and $(1+1)$-dimensional TQFTs can be seen as follows: 
\begin{enumerate}
\item\label{item:3} The circle $S^1$ is the only closed and connected $1$-manifold. A closed $1$-manifold is then a disjoint union of circles. A TQFT associates a vector space $V$ to each $S^1$ and the tensor product of vector spaces to a disjoint union of circles, denoted by $\coprod S^1$. 
\item\label{item:4} Let $\coprod_a S^1$ and $\coprod_b S^1$ be two cobordant $1$-manifolds. Let us denote the cobordism by $\zeta: \coprod_a S^1 \to \coprod_b S^1$. A dual of this map gives $\zeta^{\prime}: \coprod_b \to \coprod_a$. The significance of this dual will become apparent soon enough. Suppose a TQFT takes (functorially) them to $\bigotimes_a V$ and $\bigotimes_b V$. Then this TQFT associates $\zeta$ to a multilinear map $\partial_{\zeta} : \bigotimes_a V \to \bigotimes_b V$. 
\item\label{item:5} All cobordisms in \textbf{$2$-Cob} can be reduced to the elementary cobordism $m^{\zeta}: S^1 \coprod S^1 \to S^1$ or its dual $\Delta^{\zeta}: S^1 \coprod S^1 \to S^1$ (these are the cases of cobordism between $1$ circle and $2$ disjoint circles) . That is, all cobordisms can be expressed using operations between $m^{\zeta}$ and $\Delta^{\zeta}$. 
\item\label{item:6} Associating a map with a disk gives rise to a unit $\eta^{\zeta}$ and counit $\epsilon^{\zeta}$.
\end{enumerate}

\begin{theorem}
\label{sec:cohom-groups-mathc-14}
A commutative Frobenius algebra uniquely determines (up to isomorphism) a $(1+1)$-dimensional TQFT.
\end{theorem}

This should be clear from the discussion above. Once again we put $V = \F_2[x]/(x^2)$. We now go back to $\mathcal{S} (D)$, the category of states of $D$. Each state object $S$ in $\mathcal{S}(D)$, as we observed, is a bunch of circles and the morphisms $S_1 \to S_2$ are surface cobordisms. With our choice of $V$, we have a TQFT describing these cobordisms. Associate $V$ to each circle in each state $S$. Suppose there are $\sharp S$ circles in $S$. Our TQFT associates $V^{\otimes \sharp S} \{n_+ - 2n_- + \sharp 1\}$ to $S$. The reason of this shifting is obvious once we see that
\begin{equation}
\label{eq:16}
\dim_q V^{\otimes \sharp S} \{n_+ - 2n_- + \sharp 1\} = q^{(n_+ - 2n_- + \sharp 1)} (q + q^{-1})^{\sharp S},
\end{equation}
and then compare with the state-sum model of $\hat{J}(D)$.

Let $i = \sharp 1 - n_-$ and $j = \deg(v) + i + n_+ - n_-$ where $v \in V^{\otimes \sharp S} \{n_+ - 2n_- + \sharp 1\}$. This $i$ is called \emph{homological grading} and $j$ is called \emph{quantum grading} or \emph{q-grading}. The names are not very important to us.

Define 
\begin{equation}
\label{eq:17}
C^{i,*}(D) \coloneqq \bigoplus_{\sharp 1 = i + n_-}^{S} V^{\otimes \sharp S} \{n_+ - 2n_- + \sharp 1\}.
\end{equation}
To turn this into a chain complex, we need a differential $\partial^i : C^{i,*}(D) \to C^{i+1, *}(D)$. For this, we go back to our discussion on TQFT. Recall that our TQFT associated $V^{\otimes \sharp S}$ (we may ignore the shifting for a while) to state $S$ and multilinear maps $\partial_{\zeta}$ to surface cobordisms $\zeta$.

The map $\partial_{\zeta}$ is defined in the following way:
\begin{enumerate}
\item\label{item:7} Let $S_1$ and $S_2$ be two state objects in $Cob \mathcal{S}(D)$. The cobordism $\zeta : S_1 \to S_2$ affects at most two circles in either $S_1$ or $S_2$. Depending on whether the morphism is fusion or bifurcation, the effect of $\partial_{\zeta}$ is either multiplication $m: V \otimes V \to V$ or comulitiplication $\Delta: V \to V\otimes V$. 
\item\label{item:8} To all the circles left untouched in the cobordism $\zeta$, the effect of $\partial_{\zeta}$ is an identity map.
\end{enumerate}

In the beginning, we chose an ordering of crossings. Using that order, every morphism in the state category $\mathcal S(D)$ is represented by a binary sequence with an asterisk. This asterisk denotes the crossing where the change of smoothing occurs. There is an appropriate functor $\mathcal{S}(D) \to Cob\mathcal{S}(D)$ which takes a binary sequence to its corresponding state ($1$-manifold) and smoothing change to its corresponding cobordism $\zeta$. We may choose to use the same notation for both smoothing change and cobordism since there is little chance of confusion. Thus, $\zeta$ is a binary sequence is an asterisk. Define sgn$(\zeta):= (-1)^{l}$ where $l$ is the number of $1$s to the left of the asterisk.

Define the map $\partial^i : C^{i,*}(D) \to C^{i+1, *}(D)$ as follows 
\begin{equation}
\label{eq:18}
\partial^i(v) = \sum_{\zeta \text{ such that Tail}(\zeta) = S}^{} \text{sgn}(\zeta) \partial_{\zeta}(v),
\end{equation}
where $v \in V_S \subset C^{i,*}(D)$. Here we used the notation $V_S$ to denote the vector space, $V^{\otimes \sharp S} \{n_+ - 2n_- + \sharp 1\}$, associated to state $S$ by our TQFT.

\begin{proposition}
\label{sec:cohom-groups-mathc-15}
The map $\partial$ is a differential, that is, $\partial^{i+1} \circ \partial^i = 0$.
\end{proposition}

To see this, first notice that without the function $\text{sgn}$, the faces of the cube (state category) commutes. This can be shown either algebraically using the definition of the maps $m$ and $\Delta$ associated with $V$ (this is achieved by breaking it down into cases), or geometrically using the fact that the cobordisms along each of the two routes in a face are same up to homemorphism and hence they induce the same linear map. Then we observe that $\text{sgn}$ occur in odd numbers on every face. This turns the commutativity into anti-commutativity.

We now have a chain complex, denoted by $Kh(D)$ 
\begin{equation}
\label{eq:19}
0 \xrightarrow{} C^{0,*} \xrightarrow{\partial^0} C^{1,*} \xrightarrow{\partial^1} C^{2,*} \xrightarrow{\partial^2} \cdots \xrightarrow{\partial^{n-1}} C^{n,*} \xrightarrow{} 0.
\end{equation}

We now go back to $Mat (Cob \mathcal{S}(D))$. Define a functor $\mathcal{F}: Mat (Cob \mathcal{S}(D)) \to Kh(D)$ sending the matrix object $\mathcal{O}^k$ to vector space $C^{k,*}$ and matrix morphism $d^k$ to differential $\partial^k$. It follows that these two categories are equivalent. This can be established by defining an appropriate (and obvious) functor $G: Kh(D) \to Mat (Cob \mathcal{S}(D))$.

Given an oriented link diagram $D$, this construction gives a Khovanov chain complex $Kh(D)$. The (co)homology groups are defined in the usual way. 
\begin{equation}
\label{eq:20}
\mathcal{H}^{i,*} = \frac{\ker \partial^i}{\text{im } \partial^{i-1}}.
\end{equation}

The graded Euler characteristic of this (co)homology is defined in the usual way.
\begin{equation}
\label{eq:21}
\chi (\mathcal{H^{*,*}}) = \sum_i^{} (-1)^i \text{ rk}(\mathcal{H}^{i,*}).
\end{equation}

The degree of the differential $\partial^i$ is one and the (co)homology groups $\mathcal{H}^{i,*}$ are finite dimensional. We conclude that rk$(\mathcal{H}^{i,*}) = \dim_q C^{i,*}$. Thus, the Euler characteristic can be directed calculated from the Khovanov chain complex without computing the (co)homology groups.

\begin{theorem}
\label{sec:cohom-groups-mathc-16}
The graded Euler characteristic of the Khovanov chain complex $Kh(D)$ is the unnormalized Jones polynomial $\hat{J}(D)$.
\end{theorem}

This follows trivially from our construction. Indeed, we may expand
\begin{align*}
  \sum_i^{} (-1)^i \dim_q C^{i,*} &= \sum_i^{} (-1)^i \sum_{\sharp 1 = i + n_-}^{} \dim_q V^{\otimes |S|}\{n_+ - 2n_- + \sharp 1\}\\
   &= \sum_{}^{} (-1)^{\sharp 1 - n_-} q^{n_+ - 2n_- + \sharp 1} (q + q^{-1})^{|S|}.
\end{align*}
Since $(-1)^{\sharp 1 - n_-} = (-1)^{\sharp 1+n_-}$ (by adding $2n_-$ to the exponent), the theorem follows.

\begin{theorem}
\label{sec:categ-hatj_lq}
The Khovanov homology groups are invariants for links.
\end{theorem}

Two diagrams of isotopic links may have different chain complexes (chain groups) but the same homology groups (up to isomorphism). An explicit and detailed construction of the chain maps inducing the corresponding isomorphisms can be found in this paper \cite{jacobsson2004invariant} by Magnus Jacobsson.

%%%%%%%%%%%%%%%%%%%%%%%%%%%%%%%%%%%%%%%%%%%%%%%%%%%%%%%

\section{Categorification of $\left< \hat\cdot \right>$}

In this section, we categorify the reduced bracket $\left< \hat\cdot \right>$ following Oleg Viro's approach \cite{OlegViro2004}. One can easily guess we begin from the state-sum model of $\left< \hat\cdot \right>$. If needed, we recast the sum into a form which can easily be categorified. We use the variable $A$ used in the original bracket skein. In each summand of the modified sum, there should a $(-1)^n$ term for some $n$ and a power of $A$.

Many of the ideas from the previous section carry over. For instance, we choose an arbitrary ordering of the crossings and represent the morphisms in the state category $\mathcal{S}(D)$ by binary numbers with an asterisk. 

The unreduced bracket polynomial $\left< \hat\cdot \right>$ admits the following state-sum form:
\begin{equation}
\label{eq:25}
\hat{\left< D \right>} = \sum_{S \in \mathcal{S}(D)}^{} A^{\sigma(S)}(-A^2 - A^{-2})^{\sharp S},
\end{equation}
where $\sigma(S) = \sharp 0 - \sharp 1$, the difference between the number of $0$-smoothings (A) and $1$-smoothings (B). The power of a binomial in the summand is not very elegant. One may expand the power and break the summand into a sum of monomials thus writing $\left< \hat\cdot \right>$ as a sum of monomials only. For that purpose, we need a definition.

An enhanced state (en-state) $s$ assigns a sign (positive or negative) to each circle in $S$. Since there are $\sharp S$ circles in $S$, there are $\sum_{k=0}^{\sharp S}\binom{\sharp S}{k}$ en-states in $S$ so that we have a total of $\sum_S^{\mathcal{S}(D)} \sum_{k=0}^{\sharp S}\binom{\sharp S}{k}$ en-states.

Expanding the binomial power, we arrive at the following en-state-sum form: 
\begin{equation}
\label{eq:26}
\hat{\left< D \right>} = \sum_{s \in \mathcal{S}^e(D)}^{} (-1)^{\sharp s} A^{\sigma(s) + 2\tau(s)},
\end{equation}
where $\tau(s)$ is the number of positive circles minus the number of negative circles in state $s$ and the sum is over all the en-states $s \in \mathcal{S}^e(D)$.

\begin{remark}
We used the notation $s$ for en-states while $\mathcal{S}^e$ denotes the set of such en-states. We could have used $S^e$ for en-state but for brevity, we stick with $s$.
\end{remark}

The model of $\left< \hat\cdot \right>$ given by Equation (\ref{eq:26}) is the one we will categorify. Define
\begin{equation}
\label{eq:27}
\mathcal{S}_{a,b}(D) \coloneqq \{ s \in \mathcal{S}^e(D) \ | \ \sigma(s) = a, \sigma(s) + 2\tau(s) = b \}.
\end{equation}
Define $C_{a,b}(D; \Z) \coloneqq \Z\mathcal{S}_{a,b}(D) $, the $\Z$-module with basis $\mathcal{S}_{a,b}(D)$. It follows that $C(D; \Z) = \bigoplus_{a,b}C_{a,b}(D; \Z)$ is a bigraded $\Z$-module. This is the model of our chain complex.

\begin{remark}
The $\Z$-module $C_{a,b}(D; \Z)$ is generated by the en-states $s$ satisfying $\sigma(s) = a, \sigma(s) + 2\tau(s) = b$ in the same way $C_n(X; \Z)$ is generated by $n$-simplexes in simplicial homology. The difference is of course the bigrading.
\end{remark}

Recall that when we build the category $\mathcal{S}(D)$ in the categorification of $V_L(t)$, the effect of the morphisms $S_1 \to S_2$ are $0 \rightsquigarrow 1$ at some crossing and identity at all other crossings. This decreases $\sigma(\cdot)$ by $2$. Thus we look at morphisms $C_{a,b}(D; \Z) \to C_{a-2,b}(D; \Z)$.

To have a morphism $C_{a,b}(D; \Z) \to C_{a-2,b}(D; \Z)$, we need to define morphisms $s \to s^{\prime}$ where $\sigma(s) = \sigma(s^{\prime}) + 2$ and $\tau(s) + 1 = \tau(s^{\prime})$. The first condition is trivially satisfied by our construction since we are only looking at $s_1 \to s_2$ with $0 \rightsquigarrow 1$ at only one crossing and identity at all other crossings. By the second condition puts more restriction on the possible forms the morphism $s_1 \to s_2$ can take. In the construction of the category $Cob \mathcal S(K)$, we saw that morphisms between states are $2$-cobordisms between compact $1$-manifolds in which either two circles fuse to form a circle (multiplication $m : V\otimes V \to V$) or a circle bifurcates into two (comultiplication $\Delta : V \to V\otimes V$). Denote a positive circle by $\KPA_+$ and negative circle $\KPA_-$.

\begin{proposition}
\label{sec:categ-left-hatcd}
The following are all the forms the morphism $\zeta : s_1 \to s_2$ can take: 
\begin{align}
\label{eq:22}
\begin{split}
  \zeta : \KPA_- \KPA_- \rightsquigarrow \KPA_- &\hspace*{3em}\zeta : \KPA_+ \rightsquigarrow \KPA_+ \KPA_+ \\
  \zeta : \KPA_+ \KPA_- \rightsquigarrow \KPA_+ &\hspace*{3em}\zeta : \KPA_- \rightsquigarrow \KPA_- \KPA_+\\
  \zeta : \KPA_- \KPA_+ \rightsquigarrow \KPA_+ &\hspace*{3em}\zeta : \KPA_- \rightsquigarrow \KPA_+ \KPA_-
\end{split} 
\end{align}
\end{proposition}

The truth about the above proposition is obvious and follows from the second condition. Notice that we cannot have $\zeta : \KPA_+\KPA_+ \rightsquigarrow \KPA_{\pm}$.

\begin{remark}
The Proposition \ref{sec:categ-left-hatcd} makes it clear why we use a Frobenius algebra $V$ to describe the morphisms $\zeta$. Recall that $V = \left< 1, x\right> $ admits $m : V\otimes V \to V$ and $\Delta : V \to V \otimes V$ such that $m$ satisfies $1 \otimes 1 \rightsquigarrow 1$, $x \otimes 1 \rightsquigarrow x$, $1 \otimes x \rightsquigarrow x$, $x \otimes x \to 0$ and $\Delta$ satisfies $x \rightsquigarrow x \otimes x$, $1 \rightsquigarrow 1\otimes x + x\otimes 1$. Associate $\KPA_-$ with $1$ and $\KPA_+$ with $x$. Associate a disjoint sum of signed circles with tensor product of corresponding generators of $V$. 
\end{remark}

Define a homomorphism $\partial_{a,b}: C_{a,b}(D, \Z) \to C_{a-2,b}(D, \Z)$ by setting 
\begin{equation}
\label{6:eq:23}
\partial_{a,b}(s) = \sum_{s^{\prime} \in \mathcal{S}_{a-2,b}}^{} (-1)^{\sharp 1^+} [s, s^{\prime}] s^{\prime},
\end{equation}
where $\sharp 1^+$ the number of $1$-smoothings to the right of (or bigger than) the asterisk (or equivalently $x(D)$ as defined below) and $[s,s^{\prime}]$ is called the incidence number which requires a definition now. Indeed, from our construction in the previous section, we do not need anything fancy here. The number $[s, s^{\prime}]$ is only to make sure that $s \in S_{a,b}$ is mapped to the correct linear combination of $s^{\prime} \in S_{a-2,b}$. As shall see in coming pages, we do not need any more restriction to make $\partial_{a,b}$ a differential for the chain complex we are building other than what has already been imposed.

Then it is very natural to define $[s, s^{\prime}]$ in this way: $[s, s^{\prime}]$ takes values from the set $\{0, 1\}$ and has value $1$ if and only if the following conditions hold: (1) the states $s$ and $s^{\prime}$ are identical except at one crossing $x(D)$ at which $s$ has $0$-smoothing and $s^{\prime}$ has $1$-smoothing, (2) every circle of $s$ not interacting with $x(D)$ keeps its sign in $s^{\prime}$, (3) $\tau(s^{\prime}) = \tau(s) + 1$.

\begin{remark}
One shortcoming of defining $\sharp 1^+$ in the way we did is that we have to prove that the homology is independent of the order of the crossings. Indeed, this is the case \cite{10.1215/S0012-7094-00-10131-7}. Oleg Viro suggested an alternative definition which bypasses this checking \cite{OlegViro2004}.
\end{remark}

\begin{theorem}
\label{sec:categ-left-hatcd-4}
The homomorphism $\partial_{a,b}$ satisfies $\partial_{a-2,b} \circ \partial_{a,b} = 0$.
\end{theorem}

This is checked by expanding $\partial_{a-2,b} \circ \partial_{a,b}(s)$ using the definition of $\partial_{a,b}$ given by Equation (\ref{6:eq:23}). The above restrictions forces this equality. Thus $\partial_{a,b}$ is a differential.

The Khovanov homology is then defined to be the homology of the chain complex we have thus constructed. 
\begin{equation}
\label{eq:24}
\mathcal{H}_{a,b}(D; \Z) = \frac{\ker\partial_{a,b}}{\text{im }\partial_{a+2,b}}.
\end{equation}

One may attempt the categorify the bracket skein relation and obtain a long exact sequence of this homology \cite{OlegViro2004}. This long exact sequence may be used to compute Khovanov homology for torus links $T(2,n)$ for $n \geq 0$.

Khovanov homology is strictly more powerful than the Jones polynomial. For instance, knots $5_1$ and $10_{132}$ have the same Jones polynomial $-q^{-7} + q^{-6} - q^{-5} + q^{-4} + q^{-2}$ but they have different homology groups. Khovanov homology also detects the unknot \cite{kronheimer2011khovanov} which is still an open problem for the Jones polynomial. It gave a much easier proof of Milnor's conjecture.